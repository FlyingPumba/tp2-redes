Como conclusión queremos mencionar que la implementación de nuestro \texttt{traceroute} nos resultó sencilla, fruto de las facilidades que nos otorgó la librería scapy a la hora de armar, enviar y recibir paquetes utilizando el protocolo \texttt{ICMP}. Sin embargo nos quedó pendiente la implementación del \texttt{traceroute} usando la ip option traceroute, cuyas ventajas sobre una implementación basado en \texttt{ICMP} es la menor cantidad de paquetes que necesita para funcionar y el hecho que establece una ruta unica al destino. Esta limitación se debió a la, practicamente, inexistente documentación de la librería scapy, razón por lo que nos resultó muy dificultoso la implementación mediante prueba e error.

Por otro lado nos parece importante mencionar, a partir de los resultados de nuestros experimentos, lo común que resultan las diversas anomalias en la red. En particular los \textit{false RTT} y los \textit{missing hops} fueron las anomalias con mayor frecuencia de aparición. Entendemos que esta problematica dificulta mapeos de la red y dificulta la detección y solución de diversos problemas.

Por último nos resultó curioso las discrepancias entre la ubicación posible de los host inferida a partir de los valores de \texttt{RTT} del mismo y la ubicación de los mismos dadas por herramientas de geolocalizacion. Concluimos que estas diferencias se deben a como estan implementadas las herramientas de geolocalización, las cuales parecería que relacionan ubicación de un host con los prefijos en su dirección IP, sabiendo que rango de direcciones pertencen a que organización.
