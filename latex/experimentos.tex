En esta sección desarrollaremos y mostraremos los resultados de los experimentos para las siguientes
universidades:

\begin{itemize}
\item Universidad de Oxford(Reino Unido)
\item Universidad de Sydney (Australia)
\item Universidad de Ghana (Africa)
\item Universidad de Hong Kong de Ciencia y Tecnologia (HKUST)
\end{itemize}

Para llevar adelante los experimentos utilizamos la herramienta descripta en la introducción. La cantidad de veces que ejecutamos nuestra versión de \texttt{traceroute} para cada universidad fue de \textbf{50}. Con esa cantidad lo que buscamos  es minimizar las oscilaciones en las rutas y tiempos fruto de balanceaos de carga de los routers.

Para identificar los saltos intercontinentales, primero calculamos la variación en los \texttt{RTT} entre cada par de hops consecutivos, $\Delta RTT_{i}$, de la siguiente manera: \(\Delta RTT_{i} = RTT_{i} - RTT_{i-1}\), donde $2 \leq i \leq$ cantidad de hops. Luego tomamos estos $\Delta RTT$ como input para la tecnica de estimacion de ouliers propuesta por Cimbala\cite{Cimbala}. Los outliers detectados por la tecnica seran nuestros candidatos a enlaces intercontinentales.
A partir de este calculo, tendremos en cuenta la detección de falsos positivos y negativos contrastando nuestra inferencia con herramientas de geolocalización de direcciones IP, (\textbf{ipinfo.io o ip2location.com}).

Por último, propondremos hipótesis para los casos de comportamiento anómalo.

\subsection{Universidad de Oxford}

Los resultados de este experimento pueden resumirse en el siguiente cuadro:

\begin{table}[H]\begin{center}
    \begin{tabular}{|c|c|c|c|c|}
    \hline
    \textbf{Hop \#} & \textbf{IP}& \textbf{RTT (ms)} & \textbf{$\Delta$ RTT (ms)} & \textbf{Ubicacion (ipinfo.io)} \\ \hline
    \texttt{1} & 192.168.0.1      &1.61       & -       & IP privada   \\ \hline
    \texttt{2} & 10.27.128.1      &9.35       & 7,74       & IP privada  \\ \hline
    \texttt{3} & 10.242.1.61      & 10.83     & 1,48       & IP privada  \\ \hline
    \texttt{4} & 208.178.195.214  & 12.40     & 1,57       & Estados Unidos   \\ \hline
    \texttt{5} & 208.178.195.213  & 10.80     & −1,6       & Estados Unidos   \\ \hline
    \texttt{6} & 67.17.99.233     & 137.77    & 126,97       & Estados Unidos    \\ \hline
    \texttt{7} & *                & -        & -       & -   \\ \hline
    \texttt{8} & *                & -        & -       & -   \\ \hline
    \texttt{9} & 212.187.139.166  & 239.16    & -       & Gran Bretaña   \\ \hline
    \texttt{10} & 146.97.33.2     & 238.03    & −1,13       & Gran Bretaña   \\ \hline
    \texttt{11} & 146.97.37.194   & 238.17    & 0,14       & Gran Bretaña    \\ \hline
    \texttt{12} & 193.63.108.94   & 245.87    & 7,7       &  Gran Bretaña \\ \hline
    \texttt{13} & 193.63.108.98   & 238.36    & −7,51       & Gran Bretaña   \\ \hline
    \texttt{14} & 193.63.109.90   & 248.00    & 9,64       & Gran Bretaña    \\ \hline
    \texttt{15} & *               & -        & -       & -   \\ \hline
    \texttt{16} & *               & -        & -       & -    \\ \hline
    \texttt{17} & 192.76.32.62    & 234.21    & -       & Oxford, Inglaterra, Gran Bretaña   \\ \hline
    \texttt{18} & 129.67.242.154  & 241.62    & 7,41       & Oxfordshire, Gran Bretaña   \\ \hline
    \end{tabular}
    \caption{Ruta Universidad de Oxford (ox.ac.uk - IP 129.67.242.154)}
\end{center}\end{table}

A continuación graficamos los \texttt{RTT} en función de los hops para analizar más facilmente las anomalías detectadas.

\begin{figure}[H]
    \centering
    \begin{tikzpicture}
        \begin{axis}[
                ybar,
                xtick=data,
                width=15cm,
                ymin = 0,
                ylabel={RTT},
                xlabel={Hop number}
            ]
            \addplot table[x=hop,y=rtt]{data/oxford.dat};
        \end{axis}
    \end{tikzpicture}
\end{figure}

Con este gráfico podemos claramente observar los saltos negativos y en los que no hubo respuesta(estos
últimos los notamos con RTT -1).\\

Daremos nuestras hipótesis sobre que casos son los que identificamos para False RTT.\\

\begin{itemize}
\item Entre 4 y 5. Asimétricos.
\item Entre 9 y 10. Asimétrico.
\item Entre 12 y 13. Asimétrico.
\item Entre 14 y 17. Lo contamos porque 15 y 16 corresponden a hops perdidos. Es Asimétrico.
\end{itemize}

Por otro lado, tenemos los que no tuvieron respuesta. En nuestro caso tenemos 7, 8, 15, 16.\\

El hops 6 pertenece a Estados Unidos pero el 9 pertenece a Gran Bretaña. Pensamos que a la hora de
establecer el enlace continental se lo hace siempre primero con dos hops que nunca darán respuesta, también se
pierden hops entre una IP de Estados Unidos y la de Gran Bretaña. Como hipótesis alternativa
podríamos pensar que esto ocurre por culpa del enlace continental, ya sea por una cuestión de
excesivo transito o por causas más puntuales sobre las configuraciones o protecciones de este.\\

En el caso de 15 y 16 ocurre algo parecido, puesto que 14 pertenece a Gran Bretaña(sin ubicar region
ni ciudad) mientras que 17 pertenece a Oxford. En este caso es más probable que se trate de un caso aislado
ya que no existe un salto continental, solamente regional.\\

También podemos observar la anomalía de False Link en los hops 4 y 5 ya que muestran como ubicación a Estados Unidos
, sin embargo los RTT de ambos no corresponden con la distancia que tuvieron que recorrer entre 3 y 4. El número 6
posee un RTT más propio del primero ubicado en Estados Unidos. Una posible explicación para esta anomalía es que
los servidores intermedios entre Argentina y Estados Unidos(posiblemente ubicados en Brasil) pertenecen a una
compañía estadounidense y nuestra herramienta de ubicación utiliza este dato para aproximar su ubicación. Es posible
que el hops 6 también sufra de esta anomalía ya que usando herramientas alternativas de geolocalización de IP nos dio
de resultado Netherlands. Creemos que esta es la más acertada ya que el salto continental que habría entre
Estados Unidos y Londres sería demasiado rápido. Sin embargo, el salto que separa a Netherlands de
Londres es demasiado grande, por lo tanto no podemos estar seguros.\\

Además tenemos identificado un posible salto continental en 208.178.195.213 - 67.17.99.233.
Sin embargo esto es incorrecto ya que ambas IP pertenecen al dominio de Estados Unidos. Aunque creemos que
el salto mostrado es correcto ya que, como mencionamos antes, el hops 6 debe ser de Netherlands en
vez de Estados Unidos.

\newpage

\subsection{Universidad de Syndney}

Los resultados de este experimento pueden resumirse en el siguiente cuadro:

\begin{table}[H]\begin{center}
    \begin{tabular}{|c|c|c|c|c|}
    \hline
    \textbf{Hop \#} & \textbf{IP}& \textbf{RTT (ms)} & \textbf{$\Delta$ RTT (ms)} & \textbf{Ubicacion (ipinfo.io)} \\ \hline
    \texttt{1} & 192.168.0.1      & 1.38    & -       & IP privada  \\ \hline
    \texttt{2} & *                & -       & -       & -   \\ \hline
    \texttt{3} & *                & -       & -       & -   \\ \hline
    \texttt{4} & *                & -       & -       & -   \\ \hline
    \texttt{5} & *                & -       & -       & -   \\ \hline
    \texttt{6} & 200.89.165.9     & 14.06   & -       & Argentina    \\ \hline
    \texttt{7} & 200.89.165.250   & 14.71    & 0.65   & Argentina   \\ \hline
    \texttt{8} & 190.216.88.33    & 16.23    & 1.52  & Ciudad de Buenos Aires, Argentina   \\ \hline
    \texttt{9} & 67.17.94.249     & 240.04  & 223.81 & Estados Unidos   \\ \hline
    \texttt{10} & *               & -       & -       & -   \\ \hline
    \texttt{11} & *               & -       & -       & -    \\ \hline
    \texttt{12} & 4.68.127.54     & 213.49  & -       & Estados Unidos   \\ \hline
    \texttt{13} & 129.250.4.250   & 229.91  & 16.48    & Colorado, Estados Unidos   \\ \hline
    \texttt{14} & 129.250.2.219   & 271.08  & 41.17   & Colorado, Estados Unidos   \\ \hline
    \texttt{15} & 129.250.7.69    & 269.29  & -1.79   & Colorado, Estados Unidos   \\ \hline
    \texttt{16} & 129.250.3.123   & 273.33  & 4.04    & Colorado, Estados Unidos    \\ \hline
    \texttt{17} & 204.1.253.166   & 269.75  & -3.58   & Colorado, Estados Unidos   \\ \hline
    \texttt{18} & 202.158.194.172 & 371.34  & 101.59   & Canberra, Australia   \\ \hline
    \texttt{19} & 113.197.15.68   & 377.51  & 6.17    & Canberra, Australia   \\ \hline
    \texttt{20} & 113.197.15.66   & 377.32  & -0.19   & Canberra, Australia   \\ \hline
    \texttt{21} & 113.197.15.152  & 418.35  & 41.03   & Canberra, Australia    \\ \hline
    \texttt{22} & 138.44.5.47     & 391.39  & -26.96  & Victoria, Australia   \\ \hline
    \texttt{23} & *               & -       & -       & -                \\ \hline
    \texttt{24} & *               & -       & -       & -   \\ \hline
    \texttt{25} & 129.78.5.8      & 446.28  & -       & Sydney, Australia   \\ \hline
    \end{tabular}
    \caption{Ruta Universidad de Sydney (sydney.edu.au - IP 129.78.5.8)}
\end{center}\end{table}

A continuación graficamos los \texttt{RTT} en función de los hops para analizar más facilmente las anomalías detectadas.

\begin{figure}[H]
    \centering
    \begin{tikzpicture}
        \begin{axis}[
                ybar,
                xtick=data,
                width=15cm,
                ymin = 0,
                ylabel={RTT},
                xlabel={Hop number}
            ]
            \addplot table[x=hop,y=rtt]{data/sidney.dat};
        \end{axis}
    \end{tikzpicture}
\end{figure}

A continuación enumeramos las anomalías encontradas:
\begin{itemize}
    \item \textit{Missing hops}: de los hops 2, 3, 4, 5, 10, 11, 23 y 24 no obtuvimos respuesta.
    \item \textit{False RTT por rutas asimétricas}: entre los hops 14-15, 16-17, 19-20, 21-22 obtuvimos valores negativos en el $\Delta RTT$, valores no consistentes. En particular el caso de los hops 21-22 es el más significativo, especulamos que en este caso la ruta de vuelta de los paquetes difiere considerablemente con respecto a la ida. Por otro lado entre los hops 9-12 también encontramos una inconsistencia considerable entre los valores de \texttt{RTT} de ambos hosts, independientemente de que no sean hosts consecutivos. Nuevamente especulamos con rutas de ida y vuelta que difieren significativamente.
    \item \textit{False RTT por MPLS routing}: la cercanía de los valores de \texttt{RTT} de los hops 14, 15, 16 y 17 y, por otro lado, los hops 18, 19 y 20, nos inducen a pensar que estos valores pueden caer dentro de la descripción de esta anomalía.
\end{itemize}

Con respecto a la detección de los enlaces intercontinentales, nuestra herramienta arroja como principal candidato al enlace entre los hops 8 y 9 ($\Delta RTT = 223.81$). El otro candidato es el enlace entre los hops 17-18 ($\Delta RTT = 101.59$).

Contrastando nuestra hipótesis contra las ubicaciones arrojadas por la aplicación de geolocalizacion, vemos que el enlace entre los hops 8 y 9 es un enlace entre Argentina y Estados Unidos, lo cual, geográficamente, no es un cambio de continente, pero si es un enlace submarino. Con respecto al enlace entre los hops 17-18 es un enlace entre Estados Unidos e Australia, lo que si representa un enlace intercontinental.

\subsection{Universidad de Ghana}

Los resultados de este experimento pueden resumirse en el siguiente cuadro:

\begin{table}[H]\begin{center}
    \begin{tabular}{|c|c|c|c|c|}
    \hline
    \textbf{Hop \#} & \textbf{IP} & \textbf{RTT (ms)} & \textbf{$\Delta$ RTT (ms)} & \textbf{Ubicacion (ipinfo.io)} \\ \hline
    \texttt{1}  & 10.27.64.1      & 14.23             & -                     & IP privada                     \\ \hline
    \texttt{2}  & 10.242.1.149    & 11.58             & -2.65                 & IP privada                     \\ \hline
    \texttt{3}  & 195.22.220.33   & 13.38             & 1.8                   & Italia                         \\ \hline
    \texttt{4}  & 195.22.220.32   & 11.97             & -1.41                 & Italia                         \\ \hline
    \texttt{5}  & 195.22.206.92   & 172.40            & 160.43                & Italia                         \\ \hline
    \texttt{6}  & 195.22.206.92   & 172.13            & 0.27                  & Italia                         \\ \hline
    \texttt{7}  & 216.6.87.202    & 164.11            & -8.02                 & Delaware, Estados Unidos       \\ \hline
    \texttt{8}  & 216.6.87.169    & 255.53            & 91.42                 & Delaware, Estados Unidos       \\ \hline
    \texttt{9}  & 216.6.57.1      & 253.17            & -2.36                 & Delaware, Estados Unidos       \\ \hline
    \texttt{10} & 66.198.70.174   & 230.14            & -23.03                & Delaware, Estados Unidos       \\ \hline
    \texttt{11} & 80.231.76.121   & 253.52            & 23.38                 & Europa                         \\ \hline
    \texttt{12} & 195.219.195.238 & 325.31            & 71.79                 & Europa                         \\ \hline
    \texttt{13} & 41.21.232.70    & 324.90            & -0.41                 & Sudafrica                      \\ \hline
    \texttt{14} & 41.204.60.149   & 327.05            & 2.15                  & Ghana                          \\ \hline
    \texttt{15} & 41.204.60.150   & 327.73            & 0.68                  & Ghana                          \\ \hline
    \texttt{16} & 197.255.127.2   & 328.85            & 1.12                  & Ghana                          \\ \hline
    \texttt{17} & 197.255.125.10  & 331.42            & 2.57                  & Ghana                          \\ \hline
    \end{tabular}
    \caption{Ruta Universidad de Ghana (ug.edu.gh  - IP 197.255.125.10)}
\end{center}
\end{table}

A continuación graficamos los \texttt{RTT} en función de los hops para analizar más facilmente las anomalías detectadas.

\begin{figure}[H]
    \centering
    \begin{tikzpicture}
        \begin{axis}[
                ybar,
                xtick=data,
                width=15cm,
                ymin = 0,
                ylabel={RTT},
                xlabel={Hop number}
            ]
            \addplot table[x=hop,y=rtt]{data/ghana.dat};
        \end{axis}
    \end{tikzpicture}
\end{figure}

En primer lugar es interesante notar que los primeros dos hops muestran IPs privadas. Aunque no encontramos una explicación clara, podemos especular que estas corresponden a dos routers de un mismo proveedor de internet, quien les asignó IPs privadas para ahorrar IPs públicas.

Pasamos a analizar las anomalias encontradas:

\begin{itemize}
  \item \textit{Loop/Cycle}: en los hops 5 y 6, que tienen la misma IP. Esto podría suceder por diversas razones:
      \begin{itemize}
        \item Debido a que el hop 5 fowardea paquetes con TTL 0.
        \item Un ciclo que termina en la IP en cuestion, y que no deja ver cual es la IP verdadera del hop 5.
      \end{itemize}
  \item \textit{False RTT por rutas asimetricas}: en los hops 2, 4, 10 y 13 por tener incremento negativo en los saltos. El único salto de estos que podría llegar a ser debido a MPLS es el del hop 13 ($\Delta RTT = -0.41$), pero no fue posible confirmarlo.
\end{itemize}

Mostramos ahora los enlaces candidatos a salto continental que arroja nuestra herramienta, contrastandolos con las ubicaciones arrojadas por la aplicación de geolocalización:

\begin{enumerate}
  \item Enlace entre los hops 4 y 5 ($\Delta RTT = 160.43$): Por la diferencias entre RTTs es claro que este enlace es un salto continental. Sin embargo, la aplicación de geolocalización nos dice que las IPs de los hops 3, 4, 5 y 6 corresponden a Italia, por lo que entonces estaríamos en un caso de falso positivo.

    Pero, dado que los $\Delta RTT$ entre los hops 3 y 4, y 5 y 6 son chicos, y que todas estas IPs tienen asociada la organización ``AS6762 TELECOM ITALIA SPARKLE S.p.A.'', concluimos que:

    \begin{itemize}
      \item Los hops 3 y 4 corresponden en realidad a routers en Argentina, con IPs asignadas a Italia.
      \item Los hops 5 y 6 corresponden efectivamente a routers en Italia, con IPs asignadas a Italia.
      \item El enlace entre los hops 4 y 5 es efectivamente un salto continental.
    \end{itemize}

  \item Enlace entre los hops 7 y 8 ($\Delta RTT = 91.42$): Por la diferencias entre RTTs es claro que este enlace es un salto continental. Sin embargo, la aplicación de geolocalización nos dice que las IPs de los hops 7 y 8 corresponden a Estados Unidos, por lo que entonces estaríamos en un caso de falso positivo.

  Pero, dado que el $\Delta RTT$ entre los hops 6 y 7 es chico, y el hop 6 corresponde a Italia, podemos concluir que:
  \begin{itemize}
    \item El hop 7 corresponde en realidad a un router en Italia, con IP asignada a Estados Unidos.
    \item El enlace entre los hops 7 y 8 es efectivamente un salto continental.
  \end{itemize}

  \item Enlace entre los hops 11 y 12 ($\Delta RTT = 71.79$): Las diferencias entre RTTs nos inducen a pensar que este enlace es un salto continental, aunque la aplicación de geolocalización no nos provee información acerca de la posible ubicación de estas IPs.

    Sin embargo, debido a que el hop 10 corresponde a Estados Unidos, y el 13 a Sudafrica concluimos que el enlace es efectivamente un salto continental, y que el hop 11 corresponde un router en Estados Unidos y el hop 12 a un router en Sudafrica.
  \item Enlace entre los hops 10 y 11 ($\Delta RTT = 23.38$): en base a lo concluido para el enlace entre los hops 11 y 12, este caso sería un falso positivo.
  \item Enlace entre los hops 9 y 10 ($\Delta RTT = -23.03$): este enlace parecería ser un falso positivo, por varias razones:
      \begin{itemize}
        \item La aplicación de geolocalización informa que ambas IPs están asignadas a Delaware, Estados Unidos.
        \item El salto es negativo, lo cual podría implicar un congestionamiento en algún router.
      \end{itemize}
  \item Enlace entre los hops 6 y 7 ($\Delta RTT = -8.02$): en base a lo concluido para el enlace entre los hops 7 y 8, este caso sería un falso positivo.
\end{enumerate}


\subsection{Universidad de Hong Kong de Ciencia y Tecnologia (HKUST)}

En este caso buscamos analizar los resultados de traceroute a una universidad de Hong Kong. Dada su ubicación en Asia, esperamos un gran aumento del RTT a medida que nos acercamos al destino, asi como una gran cantidad de Hops y uno o varios saltos continentales.

Los resultados de este experimento pueden resumirse en el siguiente cuadro:

\begin{table}[H]\begin{center}
    \begin{tabular}{|c|c|c|c|c|}
    \hline
    \textbf{Hop \#} & \textbf{IP} & \textbf{RTT (ms)} & \textbf{$\Delta$ RTT (ms)} & \textbf{Ubicacion (IP2Location)} \\ \hline
    \texttt{1} &  192.168.1.1 & 0.36        &  -        & Local \\ \hline
    \texttt{2} &  200.3.60.192 & 12.24      & 11.88     & Buenos Aires \\ \hline
    \texttt{3} &  181.88.108.18 & 16.01     & 3.77      & Buenos Aires \\ \hline
    \texttt{4} &  190.225.252.166 & 13.85   & -2.16     & Buenos Aires \\ \hline
    \texttt{5} &  195.22.220.213 & 11.73    & -2.11     & Buenos Aires \\ \hline
    \texttt{6} &  195.22.219.3 & 38.76      & 27.02     & Italia \\ \hline
    \texttt{7} &  195.22.219.3 & 38.63      & -0.13     & Italia \\ \hline
    \texttt{8} &  149.3.181.65 & 87.87      & 49.24     & Brazil \\ \hline
    \texttt{9} &  129.250.2.227 & 208.21    & 120.33    & Nueva York, US \\ \hline
    \texttt{10} & 129.250.4.13 & 281.09     & 72.87     & Washington, US \\ \hline
    \texttt{11} & 129.250.2.38 & 422.20     & 141.11    & Tokyo, Japón \\ \hline
    \texttt{12} & 129.250.5.134 & 429.55    & 7.34      & Hong Kong \\ \hline
    \texttt{13} & 129.250.6.115 & 432.51    & 2.96      & Hong Kong \\ \hline
    \texttt{14} & 203.131.246.154 & 452.04  & 19.52     & Hong Kong \\ \hline
    \texttt{15} & 115.160.187.110 & 444.46  & -7.57     & Hong Kong \\ \hline
    \texttt{16} & 202.130.98.102 & 440.41   & -4.05     & Hong Kong \\ \hline
    \texttt{17} & 203.188.117.130 & 440.65  & 0.24      & Hong Kong \\ \hline
    \texttt{18} & 202.14.80.153 & 443.62    & 2.97      & Hong Kong \\ \hline
    \texttt{19} & 143.89.14.2 & 448.84      & 5.21      & Hong Kong \\ \hline
    \end{tabular}
    \caption{Ruta Universidad de Hong Kong de Ciencia y Tecnologia (www.ust.hk  - IP 143.89.14.2)}
\end{center}
\end{table}

A continuación graficamos los \texttt{RTT} en función de los hops para analizar más facilmente las anomalías detectadas.

\begin{figure}[H]
    \centering
    \begin{tikzpicture}
        \begin{axis}[
                ybar,
                xtick=data,
                width=15cm,
                ymin = 0,
                ylabel={RTT},
                xlabel={Hop number}
            ]
            \addplot table[x=hop,y=rtt]{data/hongkong.dat};
        \end{axis}
    \end{tikzpicture}
\end{figure}

Como podemos ver, a gran escala los \texttt{Round Trip Times} parecen obedecer nuestra suposición de que iba a escalar fuertemente a medida que nos acercamos al destino. La cantidad de hops en total fue 19, algo bastante grande y lógico dada la distancia entre el origen (Argentina) y el destino (Hong Kong).

La aplicación de geolocalización utilizada en este caso, ip2location.com, estipula varios saltos continentales, siendo los mismos:
\begin{itemize}
\item Argentina (Sudamerica) - Italia (Europa) entre hops 5 y 6.
\item Italia (Europa) - Brazil (Sudamerica) entre hops 7 y 8.
\item Brazil (Sudamerica) - Estados Unidos (Norteamerica) entre hops 8 y 9.
\item Estados Unidos (Norteamerica) - Japón (Asia) entre hops 10 y 11.
\end{itemize}

Nuestra propia herramienta para predecir saltos continentales detecto el salto entre Estados Unidos y Hong Kong (hops 10 a 11). Sin embargo existe claramente
una anomalía de False Link en los de Italia, ya que la diferencia de RTTs que hay entre estas y las de Argentina son demasiado pequeñas. Esto podría deberse
a que los servidores a los que pertenecen están hosteados en Italia, aunque se ubiquen físicamente en otro lugar(posiblemente Brazil). Entonces solo existe
un salto continental y es el que detecto nuestra herramienta.

Con respecto a las otras anomalías, nos sorprendió no detectar Missing Hops en ningún momento. Entre los hops 1 a 3, 11 a 13 y 16 a 19, los RTTs incrementaron gradualmente. Sin embargo tenemos las siguientes anomalías:

\begin{itemize}
\item \texttt{False RTTs}: entre los hops 3 y 4, 4 y 5, 14 y 15, 15 y 16. Esto se puede deber a rutas asimétricas que hacen que la diferencia entre los RTTs de hops sucesivos sea negativa porque en el segundo hop, el traceroute vuelve por una ruta más rápida.
\item \textit{Loop/Cycle}: entre los hops 6 y 7, esto se puede deber a que hay un ciclo en la topología y cuando el TTL es 7, el traceroute toma una ruta para llegar al mismo router que con TTL 6 que pasa por un router más que cuando el RTT es 6 pero llegan al mismo router porque hay un ciclo donde una ruta es más grande que la otra.
\end{itemize}
